\documentclass[a4paper,10pt]{article}
\usepackage{djs_cv}

\begin{document}

\cvtitle{Daniel Stonier}

\vspace{-2.5em}

\begin{center}
  \textbf{Technology/Solutions Developer} \\
  \begin{small} October 2016 \end{small}
\end{center}


\vspace{-2.5em}

\begin{cvsection}{Profile}

\vspace{-2em}

% \kofamily

\rmfamily

\begin{small} \textcolor{maroon}{\textbf{Contact Details}} \hfill \textcolor{maroon}{\textbf{Relevant Links}} \end{small}  \\ 
\begin{small} (A) \textko{서울시 광진구 군자동 356-3. 우편번호 143-840} \hfill \href{https://snorriheim.atlassian.net/wiki/display/~snorri/Portfolio}{Portfolio}\footnote{\href{https://snorriheim.atlassian.net/wiki/display/~snorri/Portfolio}{https://snorriheim.atlassian.net/wiki/display/~snorri/Portfolio}} \end{small} \\  
\begin{small} (A) South Korea, Seoul, Gwangjin-gu, Gunja-dong, \#356-3, post code 143-840. \hfill \href{https://www.linkedin.com/in/daniel-stonier-95b37522}{LinkedIn}\footnote{\href{https://www.linkedin.com/in/daniel-stonier-95b37522}{https://www.linkedin.com/in/daniel-stonier-95b37522}} \end{small}  \\ 
\begin{small} (E) d.stonier@gmail.com \hfill \href{https://github.com/stonier}{Github}\footnote{\href{https://github.com/stonier}{https://github.com/stonier}} \end{small} \\  
\begin{small} (P) +82-10-5400-3296 \hfill \href{http://blog.yujinrobot.com/}{Blog - Yujin Robot}\footnote{\href{http://blog.yujinrobot.com/}{http://blog.yujinrobot.com}} \end{small} \\
\begin{small} (S) Australian, married with two children \end{small} \\

  \vspace{-1.5em}

  I enjoy solving problems that reside in the space where algorithms, programming and technology converge on the real world. The reward of watching technology come alive has had a great part in motivating the direction of my career towards a better understanding of how to make this happen. From a mathematics PhD to company as an algorithms developer, control engineer, software architect, product manager and innovation team lead, these have all contributed to understanding a significant part of the pipeline involved in the development of technology solutions. My role is more specifically suited to that of a lead software/technology architect or developer with enough experience (and a significant interest) in product and business development to be able to participate in the bigger picture.

\end{cvsection}

\begin{cvsection}{Work History}
  \raggedright
  \cvheader{\href{http://www.yujinrobot.com/}{Yujin Robot Co. Ltd}}{Seoul, Korea}
  \cvheaderinfo{Innovation Team Co-Leader}{2014-2016}
    \cvitem{Preparation}{formed a proposal and iterated with company executives a restructured plan for Yujin's R\&D}
    \cvitem{Integration}{kickstarted the \href{http://inno.yujinrobot.com/}{innovation team}\footnote{\href{http://inno.yujinrobot.com/}{http://inno.yujinrobot.com/}} within the existing company infrastructure as a spin along}
    \cvitem{Human Resources}{hiring for a varied team (hardware, software, business, korean \& international) of 10-12 people}
    \cvitem{Business Ideation}{work with a bizdev \& product manager to finding new business and product ideas (e.g. \href{http://gocart.yujinrobot.com/}{gocart}\footnote{\href{http://gocart.yujinrobot.com/}{http://gocart.yujinrobot.com/}})}
    \cvitem{Networking}{successfully connected with partners internationally and locally}
    \cvitem{Prototypes}{rapid design and iteration on product ideas (seeing is believing!)}
    \cvitem{Shows \& Field Tests}{prepared systems for successful shows and field tests at client facilities}
    \cvitem{Software Vertical}{design and co-ordinate software development from robot firmware through to server and web applications}
    \cvitem{Service Design}{the technology voice in service design, also minimise complexity transmitted by software design to the user(s)}
    \cvitem{Quality Assurance}{manage testing, continuous integration, deployment for generating reliable software}
    \cvitem{Collaborate}{design, document, package, deliver and issue track the software for other teams and external partners}
    \cvitem{Navigation}{rebuild a multi-floor navigation stack around Yujin's cleaning robot vision slam appropriate for a large logistics robot}
    \cvitem{Behaviour Trees}{built a comprehensive implementation to handle decision making and logic inside the robot}
    \cvitem{\href{http://roscon.ros.org/2016}{ROSCon 2016}}{local chair on the organising committee for the conference}
  \cvheaderinfo{Kobuki Project Lead}{2012-2016}
    \cvitem{Goal}{a robotics research platform to connect us with the intntl dev. community to stimulate hiring \& networking potential}
    \cvitem{Successes}{thousands of users, increasing sales every year \& thriving groups around spin-off platforms (\href{http://wiki.ros.org/Robots/TurtleBot}{turtlebot}\footnote{\href{http://wiki.ros.org/Robots/TurtleBot}{http://wiki.ros.org/Robots/TurtleBot}}, \href{https://www.autonomous.ai/deep-learning-robot}{deepbot} \& \href{http://www.quanser.com/products/qbot2}{qbot})}
    \cvitem{Product Management}{co-ordinate software, hardware, manufacturing and marketing for the product, \href{http://kobuki.yujinrobot.com}{kobuki}\footnote{\href{http://kobuki.yujinrobot.com/}{http://kobuki.yujinrobot.com/}}}
    \cvitem{Business}{linked and launched kobuki in association with the \href{http://wiki.ros.org/Robots/TurtleBot}{turtlebot} platform from \href{http://www.osrfoundation.org/}{OSRF}}
  \cvheaderinfo{Lead Developer}{2011-13}
    \cvitem{Lead Roles}{co-ordinating and assisting where needed in Yujin's control team of approximately 10 people}
    \cvitem{Human Resources}{hiring of new international engineers}
    \cvitem{New Projects}{propose and work with local and international groups on various government funded projects}
    \cvitem{Algorithms}{experimental design/testing of vision slam systems using nonlinear optimisation}
    \cvitem{Robotics in Concert}{lead designer for a multirobot-device framework with OSRF, University of Texas et. al.}
    \cvitem{Software Manager}{manage the open and closed source software (direction, development and quality control)}
    \cvitem{\href{http://www.ros.org/}{ROS} Contributions}{design \& dev. of fundamental parts of the robot operating system (build, comms, platforms etc)}
    \cvitem{\href{http://roscon.ros.org/}{ROSCon}}{speaker at the annual ROS conferences}
    \cvitem{Academia}{panel member for PhD defences both locally and internationally}
  \cvheaderinfo{Cleaning Robot Product Engineer}{2009-10}
    \cvitem{Goal}{work with our partner (Philips) and their management team to bring a vision-based robot to market}
    \cvitem{Successes}{made it through the Philips project incubation trials (only 10\% succeed) and delivered a product to market}
    \cvitem{Vision Slam}{rebuild an experimental filtering algorithm from academia for vision based navigation with a focus on product}
    \cvitem{Stabilisation}{ensure the automatic navigation is failure free - i.e. for all robots, all environments and all times of the day}
    \cvitem{Quality}{worked with the Philips team to ensure market readiness (incl. 4 weeks of testing on site in the Netherlands)}
  \cvheaderinfo{Senior Control Engineer}{2007-10}
    \cvitem{Control Software Management}{integrate Yujin's control systems framework with the ROS framework}
    \cvitem{Control Software Development}{visual servoing, manipulation, embedded systems, path planning, firmware motor systems}
  \cvheader{\href{http://rit.kaist.ac.kr/home}{Korean Advanced Institute of Science and Technology [KAIST]}}{Daejeon, South Korea}
  \cvheaderinfo{Postdoctoral Fellow (Robot Intelligence Lab)}{2005-6}
    \cvitem{Research Areas}{nonlinear control for omni-navigation, postural balance for humanoids and fuzzy path planning}
    \cvitem{Supervision}{assisted lab members with theoretical mathematics and directly guided three postgrad students}
\end{cvsection}

\begin{cvsection}{Open Source Experience}
 \raggedright
  \begin{djs_itemize}
    \item \textbf{\href{http://wiki.ros.org/Robots/TurtleBot}{Turtlebot}\footnote{\href{http://wiki.ros.org/Robots/TurtleBot}{http://wiki.ros.org/Robots/TurtleBot}}} [2012-15] - responsible for \& worked with Willow/OSRF to maintain \& extend the turtlebot stack for ros groovy \begin{small}$\rightarrow$\end{small} indigo
    \item \textbf{\href{http://wiki.ros.org/rosjava?distro=indigo}{RosJava}\footnote{\href{http://wiki.ros.org/rosjava?distro=indigo}{http://wiki.ros.org/rosjava?distro=indigo}}} [2012-14] - worked with Damon Kohler to maintain \& provide mature catkin/gradle/maven interfaces for indigo
    \item \textbf{\href{https://github.com/ros-windows}{Ros on Windows}}\footnote{\href{https://github.com/ros-windows}{https://github.com/ros-windows}} [2010-12] - a minimal environment for company \& factory software, also turned out to be useful for others
    \item \textbf{Catkin} [2010] - worked with devs at Willow/OSRF in the design phase to enable native cross-compilation/platform build tools.
    \item \textbf{Maintainer} [2008-16] - maintain a great many open source packages, examples include \href{https://github.com/stonier/sophus}{sophus}, \href{https://github.com/plasmodic/ecto}{ecto}, kobuki, turtlebot...
    \item \textbf{Contributions} [2008-16] - patches for ros on arm, opencv, redmine and many robotics packages in the ros community
    \item \textbf{Projects} [2008-16] - \href{http://wiki.ros.org/ecl}{ecl (wiki)}, \href{https://github.com/stonier/ecl_core}{ecl (code)}, \href{https://github.com/stonier/py_trees_suite}{py\_trees}, \href{https://github.com/robotics-in-concert/rocon_multimaster}{ros multimaster}, \href{https://github.com/robotics-in-concert/rocon_tools}{rocon tools}, \href{https://github.com/stonier/message_multiplexing}{mm} \& many smaller packages for robotics
  \end{djs_itemize}
\end{cvsection}

\begin{cvsection}{General Skills}
 \raggedright
  \begin{djs_itemize}
    \item \textbf{Product Development} - have particpated in various parts of the product pipeline for several projects
    \begin{djs_itemize}
      \item \textit{GoCart} - prototype robotic platform and business solution for autonomous logistics.
      \item \textit{Kobuki} - a mobile research base.
      \item \textit{Turtlebot 2} - a mobile research platform and software environment.
      \item \textit{iClebo HomeRun} - a vision based cleaning robot developed in co-operation between Yujin \& Philips.
    \end{djs_itemize}
    \item \textbf{Bridging Experimental to Product} - taking the new and shiny and incorporating it into product development.
    \begin{djs_itemize}
      \item \textit{Research} - from academic paper to a level required for a robust product (vision slam for cleaning robots).
      \item \textit{Software Quality} - understanding where and how to draw the line between rapid delivery and long term sustainability
      \item \textit{Timing} - a good track record in selectively introducing or blocking new technologies which become vindicated later and effectively gave us a head start (Linux, Eigen, ROS, Turtlebot 2, Web Tools...).
    \end{djs_itemize}
    \item \textbf{Communication}
    \begin{djs_itemize}
      \item \textit{C-Level} - worked with our c-level executives to integrate an 'innovation team' into a korean company.
      \item \textit{Inter-Company} - connections with groups at developer and executive level to form partnerships (Willow, OSRF, ScanBox, ...)
      \item \textit{Small Teams} - have worked in teams of up to 15 people with very mixed skills and cultures (typical robotics team). 
    \end{djs_itemize}
    \item \textbf{Solutions}
    \begin{djs_itemize}
      \item \textit{Nailing It} - never happy until I nail it, which lets me dig deeper until the solution is found, or the cause properly identified.
      \item \textit{Decision Making} - having both research and product background makes it easier to make judgement calls on direction.
    \end{djs_itemize}
    \item \textbf{Software}
    \begin{djs_itemize}
      \item \textit{The Vertical} - developed, mentored and co-ordinated on software from firmware to the web, from design to deployment.
      \item \textit{Math Background} - easy to pick up new research from many fields and develop/co-ordinate theoretical development.
    \end{djs_itemize}
    \item \textbf{Languages}
    \begin{djs_itemize}
      \item \textit{English} - native.
      \item \textit{Korean} - intermediate, currently living, studying and working in korea (9+years).
    \end{djs_itemize}
  \end{djs_itemize}
\end{cvsection}

\vspace{2em}

\begin{cvsection}{Technical Skills}
 \raggedright
  \begin{djs_itemize}
    \item \textbf{C/C+} [\textit{expert}] : mathematical algorithms, templates, metaprogramming, \href{https://github.com/Itseez/opencv/pull/3194}{qt}, \href{https://github.com/stonier/ecl_core}{library development}, lower and higher level control (motor systems, navigation, manipulation), cross-compiling, bare metal embedded, opengl, many others.
    \item \textbf{Python} [\textit{expert}] - scripts, graphical interfaces, libraries and frameworks, typically for robots and robot servers.
    \item \textbf{CMake} [\textit{expert}] - used extensively in cross-platform projects and helped design ROS's \href{https://github.com/ros/catkin}{catkin} build environment.
    \item \textbf{Open Source} [\textit{expert}] - developed, contributed to or co-maintained many \href{https://github.com/stonier}{open source projects}.
    \item \textbf{Issue Tracking \& CI} [\textit{expert}] - responsible for issue tracking/continuous integration/deployment operations (e.g. jenkins).
    \item \textbf{Linux} [\textit{expert}] - custom embedded distros, real-time development, kernel building, server administration and teaching my wife.
    \item \textbf{SysAdmin} [\textit{intermediate}] - performed software related tasks for a company's needs (e.g. wiki, code services, etc).
    \item \textbf{Matlab Programming} [\textit{intermediate}] - used while at university to prototype/simulate ideas.
    \item \textbf{Web Programming} [\textit{beginner}] - supervisory experience of a web development team, enough to understand and appreciate differences in robot/web development (rest api's, javascript frameworks, docker).
    \item \textbf{Java/Android} [\textit{beginner}] - java, gradle (e.g. \href{https://github.com/rosjava}{rosjava}) and simple android interfaces.
    \item \textbf{Document Editing} [\textit{anything}] - latex, markdown, wikis, web presentations.
  \end{djs_itemize}
\end{cvsection}

\begin{cvsection}{Education}
  \raggedright
  \cvheader{\href{http://www.uq.edu.au}{University of Queensland}}{Brisbane, Australia}
  \cvheaderinfo{Bachelor of Engineering}{1991-94,2004} 
  \cvitem{Electrical Engineering}{first class honours.}
  \cvitem{Hons. Thesis}{hardware/software implementation of a robotic vision system.}
  \cvheaderinfo{Bachelor of Science}{1991-94}
  \cvitem{Mathematics}{honours stream.}
  \cvheader{\href{http://www.deakin.edu.au}{Deakin University}}{Melbourne, Australia}
  \cvheaderinfo{Doctor of Philosophy (Mathematics)}{1996-2002}
  \cvitem{Cocycle Theory}{analysis of attractors in non-autonomous dynamical systems.}
  \cvitem{Non-Autonomous Stability}{extending \& integrating cocycle and classical theories.}
  \cvitem{Numerical Analysis}{perturbations of automonous systems to non-autonomous systems.}
  \cvheader{\href{http://www.cqu.edu.au}{Central Queensland University}}{Rockhampton, Australia}
  \cvheaderinfo{Bachelor of Science (Hons.)}{1995}
    \cvitem{Mathematics}{first class honours.}
    \cvitem{Hons. Thesis}{sliding mode control of robotic manipulators.}
   \cvheader{\href{http://yeppoonshs.eq.edu.au/wcmss/}{Yeppoon State High School}}{Yeppoon, Australia}
   \cvheaderinfo{Junior and Senior Dux}{1988,1990}
\end{cvsection}

\vspace{2em}

\begin{cvsection}{Non-Curricular Interests}
  \raggedright
  \vspace{-0.5em}
  \begin{djs_itemize}
    \item \textbf{Family} - married to a korean with two children, struggling to keep up with their korean!
    \item \textbf{Cycling} - these days just to maintain fitness and enjoy some MTB tours, previously road and track racing.
    \item \textbf{Running/Swimming} - again, also to just maintain fitness and my sanity from day to day.
    \item \textbf{Chess/Squash} - whenever I can find an opponent and whenever I can find the time!
  \end{djs_itemize}
\end{cvsection}

\begin{cvsection}{References}
  \raggedright
  \cvheader{Sam Park}{CTO, Yujin Robot}
    \cvitem{Capacity Known}{R\&D Group Head, direct supervisor}
    \cvitem{Email}{sampark@yujinrobot.com}
    \cvitem{Phone}{+82-10-6295-3057}
  \cvheader{\href{https://www.linkedin.com/in/marcusliebhardt/en}{Marcus Liebhardt}\footnote{\href{https://www.linkedin.com/in/marcusliebhardt/en}{https://www.linkedin.com/in/marcusliebhardt/en}}}{Innovation Team Co-Leader, Yujin Robot}
    \cvitem{Capacity Known}{Team Co-Leader}
    \cvitem{Email}{ich@marcusliebhardt.de}
    \cvitem{Phone}{+82-010-3322-0566}
  \cvheader{\href{http://www.osrfoundation.org/team/brian-gerkey/}{Brian Gerkey}\footnote{\href{http://www.osrfoundation.org/team/brian-gerkey}{http://www.osrfoundation.org/team/brian-gerkey}}}{CEO, Open Source Robotics Foundation}
    \cvitem{Capacity Known}{Collaborative contact on various projects, co-organised ROSCon 2016}
    \cvitem{Email}{gerkey@osrfoundation.org}
    \cvitem{Phone}{+1-650-450-9682}
  \cvheader{\href{https://www.linkedin.com/in/naveen-kuppuswamy-5913808}{Naveen Kuppuswamy}\footnote{\href{https://www.linkedin.com/in/naveen-kuppuswamy-5913808}{https://www.linkedin.com/in/naveen-kuppuswamy-5913808}}}{Robotics Researcher, TRI}
    \cvitem{Capacity Known}{Former student, long time personal acquaintance}
    \cvitem{Email}{naveen.sk@gmail.com}
    \cvitem{Phone}{+41-764073803}
\end{cvsection}

\end{document}
